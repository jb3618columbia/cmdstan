\chapter{Overview}

\noindent
\CmdStan is the command-line interface for \Stan. The next two
chapters describes tools that are built as part of \CmdStan
installation: \code{stanc} and \code{print}. The process of building a
\CmdStan executable from a \Stan program is as follows:
%
\begin{enumerate}
  \item A \Stan program is written to file with a \code{.stan}
    extension.
  \item \code{stanc} is used to translate the \Stan program into a
    \Cpp file. This \Cpp file is not a full program that can be
    compiled to executable directly, but a translation from the \Stan
    language into a \Cpp concept. Each interface will generate identical
    \Cpp for the same \Stan program.
  \item A \CmdStan exectuable is generated from the \CmdStan source
    and the generated \Cpp. Each \Stan program will have its own
    \CmdStan executable. The options to the \CmdStan executable are
    described in \refchapter{stan-cmd}.
\end{enumerate}


\section{Building the \CmdStan Tools}\label{build.section}

The easy way to build \CmdStan is through the use of make. From a
command line window, type:
%
\begin{quote}
\begin{Verbatim}[fontshape=sl,fontsize=\small]
> cd <cmdstan-home>
> make build
\end{Verbatim}
\end{quote}
%
This will build both \code{stanc} and \code{print}. If your computer
has multiple cores and sufficient ram, the build process can be parallelized
by providing the \code{-j} option. For example, to build on 2 cores, type:
%
\begin{quote}
\begin{Verbatim}[fontshape=sl,fontsize=\small]
> cd <cmdstan-home>
> make -j2 build
\end{Verbatim}
\end{quote}
%
\emph{Warning:} \ The \code{make} program may take 10+ minutes and
consume 2+ GB of memory to build \CmdStan.  Compiler warnings,
such as \code{uname:~not found}, may be safely ignored.
